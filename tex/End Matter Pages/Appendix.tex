\phantomsection
\setstretch{1.5}
\justify
\fontsize{14}{16}\selectfont
\setlength{\parindent}{0pt}
\chapter*{Dodatek} 
\addcontentsline{toc}{chapter}{\textnormal{Dodatek}}
\fontsize{12}{14}\selectfont
\textbf{Funkcje pomocnicze}:

\vspace{\baselineskip}

Oblicza średni błąd kwadratowy między wartościami prawdziwymi i przewidywanymi.
\begin{pythoncode}
def rmse(y_true, y_pred):
    return np.sqrt(mean_squared_error(y_true, y_pred))
\end{pythoncode}

\vspace{\baselineskip}
\hspace{1.5cm} Oblicza całkowity czas pozostały w minutach do ukończenia treningu pozostałych modeli.
Oblicza się go, sumując średni czas potrzebny każdemu modelowi na ukończenie treningu każdej cechy.
Następnie tę średnią mnoży się przez liczbę cech, których model jeszcze nie trenował.
\begin{pythoncode}
def calculate_total_time_left(Results):
    first_column = list(Results.keys())[0]
    total_time_per_model = {model: [] for model in Results[first_column]}

    for feature in Results:
        for model_name in Results[feature]:
            time_to_add = Results[feature][model_name]["time_taken"]
            if time_to_add is None:
                continue
            total_time_per_model[model_name].append(time_to_add)

    total_time_left = sum(
        (
            np.mean(fit_time) * (len(Results.keys()) - len(fit_time))
            if len(fit_time) > 0
            else 0
        )
        for fit_time in total_time_per_model.values()
    )
    return round(total_time_left, 2)
\end{pythoncode}


\vspace{\baselineskip}
Liczy liczbę kombinacji podanych hiperparametrów.
\needspace{4\baselineskip}
\begin{pythoncode}
def count_combinations(element: tuple[type, dict]):
    model, options = element
    model_name = model.__name__
    options_combination = np.prod([len(v) for v in options.values()])
    return model_name, int(options_combination)
\end{pythoncode}

\vspace{\baselineskip}
\needspace{4\baselineskip}
Pomocnicze typy danych.

\begin{pythoncode}
from collections import namedtuple

Data_transformed = namedtuple("Data", ["X", "y"])
Data_splitted = namedtuple("Data", ["X_train", "X_test", "y_train", "y_test"])
Grid_result = namedtuple("Grid_result", ["instance", "r2_test", "rmse_test"])
\end{pythoncode}

\vspace{\baselineskip}
Pomocniczy dekorator do wyznaczania jak długo zajęło funkcji na wykonanie zadania. Czas zwraca w minutach.
\needspace{4\baselineskip}
\begin{pythoncode}
import time

def performance_decorator(func):
    total_time = 0

    def wrapper(*args, **kwargs):
        start_time = time.perf_counter()

        result = func(*args, **kwargs)

        time_took_seconds = time.perf_counter() - start_time

        time_took_minutes = round(time_took_seconds / 60, 2)

        nonlocal total_time
        total_time += time_took_minutes
        print(f"Total time: {round(total_time, 2)}[min]")

        return result, time_took_minutes

    return wrapper
\end{pythoncode}

\vspace{\baselineskip}
Wymagane biblioteki do obsługi skryptu.
\needspace{4\baselineskip}
reqs.txt
\begin{pythoncode}
emmet==2018.6.7
joblib==1.4.2
matminer==0.9.3
numpy==2.2.2
pandas==2.2.3
python-dotenv==1.0.1
scikit_learn==1.4.2
mp_api==0.45.1
\end{pythoncode}