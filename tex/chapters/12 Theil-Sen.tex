\phantomsection
\setstretch{1.5}
\justify
\fontsize{14}{16}\selectfont
\setlength{\parindent}{0pt}
\needspace{4\baselineskip}
\section*{8. Regresja Theil-Sena}
\addcontentsline{toc}{section}{8. Regresja Theil-Sena}
\fontsize{12}{14}\selectfont
\vspace{-1.0em}

\hspace{1.5cm} Regresja Theil-Sena to metoda statystyczna stosowana do szacowania nachylenia liniowej trendu między dwiema zmiennymi. Jest ona szczególnie skuteczna w przypadku występowania anomalii/wartości odstających.

\phantomsection
\setstretch{1.5}
\section*{Główne cechy  \cite{url_theil_sen_robust_regression, url_theilsen_regression_and_estimator}}
\vspace{-1.0em}

\begin{itemize}

\item Regresja Theil-Sena jest mniej wrażliwa na wartości odstające w porównaniu do zwykłej regresji liniowej. 


\item Nachylenie linii regresji jest określane poprzez obliczenie nachyleń między wszystkimi parami punktów danych, a następnie wzięcie mediany tych nachyleń. 
Po oszacowaniu nachylenia, przecięcie można obliczyć, używając mediany reszt lub dowolnego punktu na dopasowanej linii.


\item To podejście, zamiast polegać na średniej, zapewnia bardziej stabilną ocenę, gdy występują wartości odstające.
\end{itemize}


\phantomsection
\setstretch{1.5}
\section*{Wzory ogólne \cite{url_TheilSenRegressor}}
\vspace{-1.0em}

Dla zbioru danych z $n$ obserwacji, nachylenie $m$ oblicza się w następujący sposób:

$$
m = \text{median}\left(\frac{y_j - y_i}{x_j - x_i}\right) \quad \text{dla każdego } i < j
$$

gdzie $(x_i, y_i)$ i $(x_j, y_j)$ są parami obserwacji.

Po określeniu nachylenia, przecięcie $b$ można obliczyć za pomocą:

$$
b = \text{median}(y_i - m x_i)
$$

\needspace{4\baselineskip}
Ostateczne (uproszczone) równanie linii regresji Theil-Sena można wyrazić jako:

$$
y = mx + b
$$

\phantomsection
\setstretch{1.5}
\section*{Zalety }
\vspace{-1.0em}


\begin{itemize}


\item Regresja Theil-Sena zapewnia wiarygodne oszacowania nawet wtedy, gdy dane zawierają wartości odstające.
\item Łatwo zinterpretować (jest to poniekąd regresja liniowa która jest dopasowana z pominięciem anomalii).
\end{itemize}

\phantomsection
\setstretch{1.5}
\section*{Ograniczenia \cite{Blunck2006, cole1989optimal, Chan2010}}
\vspace{-1.0em}

\begin{itemize}

\item Głównym wyzwaniem dla estymatora Theil-Sena jest jego $O(n^2)$ złożoność wynikająca z obliczania nachyleń między wszystkimi parami punktów. Istnieją jednak bardziej wydajne metody, osiągające $O(n \log n)$ złożoność czasową przy użyciu deterministycznych lub losowych algorytmów. W modelach wykorzystujących operacje na bitach,  możliwe są jeszcze szybsze implementacje z losowym oczekiwanym czasem $O(n \sqrt {\log n})$.

\item Podobnie jak inne metody regresji liniowej, Theil-Sen zakłada liniową zależność między zmiennymi, co może czynić ten model nie elastycznym.
\end{itemize}

\phantomsection
\setstretch{1.5}
\section*{Zastosowania \cite{Fernandes2005, hirsch1982techniques, vaidyanathan2005comprehensive, akritas1995theil, romanic2014long} }
\vspace{-1.0em}

\hspace{1.5cm} Model Theil-Sena jest stosowany w różnych dziedzinach ze względu na jego odporność na wartości odstające oraz wartości 'cenzorowane' (astronomia). W biofizyce jest stosowany do oszacowania powierzchni liści. Jest preferowany w przypadku sezonowych danych środowiskowych, takich jak jakość wody, ze względu na precyzję w przypadku asymetrycznych("skewed") danych. W informatyce pomaga identyfikować trendy starzenia się oprogramowania, a w meteorologii ocenia długoterminowe trendy wiatru.



\phantomsection
\setstretch{1.5}
\needspace{4\baselineskip}
\section*{Użyte parametry dla GridSearchCV \cite{url_TheilSenRegressor, url_grid_search}}
\vspace{-1.0em}

\begin{itemize}
\setlength\itemsep{-0.5em}
\item max\_iter: określa maksymalną liczbę iteracji algorytmu optymalizacji.
\item max\_subpopulation: określa maksymalną liczbę podzbiorów używanych w procesie dopasowania.
\item fit\_intercept: parametr logiczny. wskazuje czy obliczyć przecięcie dla tego modelu.
\item tol: ustawia tolerancję dla kryteriów zatrzymania w procesie optymalizacji.
\end{itemize}

\noindent\makebox[\linewidth]{\rule{\paperwidth}{0.4pt}}
