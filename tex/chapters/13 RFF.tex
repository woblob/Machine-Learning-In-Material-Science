\phantomsection
\setstretch{1.5}
\justify
\fontsize{14}{16}\selectfont
\setlength{\parindent}{0pt}
\needspace{4\baselineskip}
\section*{9. Random Forest Regressor \cite{alma991000280759708832}}
\addcontentsline{toc}{section}{9. Random Forest Regressor}
\fontsize{12}{14}\selectfont
\vspace{-1.0em}

\hspace{1.5cm} Random Forest Regressor to metoda uczenia się zespołowego, która łączy wiele drzew decyzyjnych w celu uzyskania dokładniejszych i bardziej solidnych prognoz dla zadań regresji. Jest częścią szerszej rodziny algorytmów Random Forest, które mogą być używane zarówno do klasyfikacji, jak i regresji.

\hspace{1.5cm} Random Forest Regressor buduje wiele drzew decyzyjnych podczas treningu. Każde drzewo jest trenowane na losowym podzbiorze danych treningowych. Takie podejście wprowadza różnorodność wśród drzew i pomaga zmniejszyć ryzyko nadmiernego dopasowania.

\hspace{1.5cm} Algorytm wykorzystuje technikę zwaną 'agregacją przykładów wstępnych' (bagging), w której każde drzewo jest trenowane na próbce "bootstrap'owej" (losowe próbkowanie z zastępowaniem)zestawu danych. Oznacza to, że każde drzewo widzi nieco inną wersję danych.

\hspace{1.5cm} Podczas dzielenia węzłów w każdym drzewie, Random Forest wybiera losowy podzbiór cech zamiast brać pod uwagę wszystkie cechy. To zwiększa różnorodność drzew i pomaga zapobiegać nadmiernemu dopasowaniu.

\hspace{1.5cm} W przypadku zadań regresji ostateczna prognoza regresora Random Forest jest zazwyczaj średnią prognoz wykonanych przez wszystkie poszczególne drzewa. Ten proces uśredniania pomaga wygładzić błędy i poprawić ogólną dokładność modelu.

\phantomsection
\setstretch{1.5}
\section*{Zalety }
\vspace{-1.0em}


\begin{itemize}

\item Random Forest posiada mniejszą podatność na nadmierne dopasowanie w porównaniu do pojedynczych drzew decyzyjnych. Zawdzięcza to swojej naturze zespołowej i mechanizmowi uśredniania. 

\item Model ten potrafi uchwycić złożone relacje między cechami a zmienną docelową bez konieczności przyjmowania założeń liniowych.

\item Dodatkowo dostarcza wglądu w znaczenie cech, pomagając zidentyfikować zmienne, które mają największy wpływ na prognozy.

\end{itemize}

\phantomsection
\setstretch{1.5}
\section*{Ograniczenia }
\vspace{-1.0em}

\begin{itemize}

\item Jest znacznie trudniejszy do zinterpretowania niż pojedyncze drzewa decyzyjne. Zespołowa natura modelu utrudnia zrozumienie, w jaki sposób tworzone są prognozy. 

\item Szkolenie wielu drzew może być intensywne obliczeniowo, szczególnie w przypadku dużych zestawów danych lub przy użyciu wielu funkcji.
\end{itemize}

\phantomsection
\setstretch{1.5}
\section*{Zastosowania \cite{lewoniewski2016quality, wecel2015modelling}}
\vspace{-1.0em}

\hspace{1.5cm} Zastosowanie w pracach naukowych: Algorytm jest stosowany w pracach naukowych, np. do oceny jakości artykułów Wikipedii.

\phantomsection
\setstretch{1.5}
\section*{Kluczowe różnice między Regresorami Random Forest a Drzewem Decyzyjnym}
\vspace{-1.0em}

\begin{itemize}

\item Regresor drzewa decyzyjnego (RDD) opiera się na pojedynczym drzewie decyzyjnym, które tworzy prognozy poprzez rekurencyjne dzielenie danych na podstawie wartości cech. W przeciwieństwie do tego, regresor lasu losowego (RLL) buduje wiele drzew decyzyjnych podczas treningu i łączy ich prognozy, aby wygenerować ostateczny wynik poprzez uśrednienie prognoz ze wszystkich drzew.

\item RDD jest bardziej podatny na przetrenowanie, szczególnie gdy może rosnąć głęboko bez ograniczeń. Z kolei RLL jest bardziej odporny na nadmierne dopasowanie dzięki uśrednianiu wielu drzew oraz losowości wprowadzanej przez 'bagging' i wybór cech.


\item Jeśli chodzi o metodę przewidywania, regresor drzewa decyzyjnego tworzy przewidywania wyłącznie na podstawie struktury pojedynczego drzewa, co może prowadzić do dużej wariancji w przewidywaniach. W przeciwieństwie do tego, regresor Random Forest zmniejsza wariancję i poprawia dokładność poprzez uśrednianie wyników wszystkich pojedynczych drzew.


\item Ze względu na charakter zespołu wielu drzew, RLL jest trudniejszy do zanalizowania niż RDD. 


\item Złożoność obliczeniowa to kolejny aspekt różniący te modele. Regresor drzewa decyzyjnego jest zazwyczaj szybszy w szkoleniu, ponieważ wymaga zbudowania tylko jednego drzewa. Natomiast regresor lasu losowego wymaga większych zasobów obliczeniowych i czasu na szkolenie ze względu na budowę wielu drzew oraz potrzebę agregacji.


\item Oba modele mogą dostarczać informacji o ważności cech; jednak metryki ważności cech generowane przez regresor lasu losowego są często uznawane za bardziej wiarygodne dzięki agregacji wyników z wielu drzew.
\end{itemize}

\phantomsection
\setstretch{1.5}
\section*{Użyte parametry dla GridSearchCV \cite{url_RFF, url_grid_search}}
\vspace{-1.0em}

\begin{itemize}
\setlength\itemsep{-0.5em}
\item n\_estimators: określa liczbę drzew w lesie.
\item max\_leaf\_nodes: ustawia maksymalną liczbę węzłów liściowych w każdym drzewie.
\item max\_depth: określa maksymalną głębokość każdego drzewa.
\item bootstrap: parametr logiczny. Wskazuje czy 'bootstrap'ować' próbki do budowania drzew. Jeśli False, cały zestaw danych jest używany do budowania każdego drzewa.
\end{itemize}
% \noindent\makebox[\linewidth]{\rule{\paperwidth}{0.4pt}}