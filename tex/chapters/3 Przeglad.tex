\phantomsection
\setstretch{1.5}
\justify
\fontsize{14}{16}\selectfont
\setlength{\parindent}{0pt}
\section*{I. Przegląd uczenia maszynowego \cite{alma991000280759708832}}
\label{sec:machine_learning_overview}
\addcontentsline{toc}{chapter}{I. Przegląd uczenia maszynowego}
\fontsize{12}{14}\selectfont
\vspace{-1.0em}



\hspace{1.5cm} Uczenie maszynowe (ML) to podzbiór sztucznej inteligencji (AI), który koncentruje się na opracowywaniu algorytmów i modeli statystycznych. Umożliwia komputerom wykonywanie zadań bez wyraźnych instrukcji. Zamiast tego systemy ML uczą się na podstawie danych i poprawiają swoją wydajność w czasie dzięki doświadczeniu. Uczenie maszynowe to narzędzie, które umożliwia komputerom uczenie się na podstawie danych i podejmowanie „świadomych” decyzji.



\phantomsection
\setstretch{1.5}
\section*{Czym jest uczenie maszynowe?}
\vspace{-1.0em}
\label{sec:what_is_ml}
\addcontentsline{toc}{section}{1. Czym jest uczenie maszynowe?}

\hspace{1.5cm} Uczenie maszynowe obejmuje trenowanie algorytmów na dużych zestawach danych w celu rozpoznawania wzorców, podejmowania decyzji i przewidywania wyników. Obejmuje różne techniki i metodologie, które można ogólnie podzielić na cztery główne typy:

\begin{itemize}
    \item \textbf{Uczenie nadzorowane}: W tym podejściu model jest trenowany na oznaczonych danych, gdzie cechy wejściowe są sparowane z prawidłowym wynikiem. Celem jest nauczenie się mapowania danych wejściowych na dane wyjściowe, tak aby po dostarczeniu nowych, niewidzianych danych model mógł przewidzieć prawidłowy wynik.
    

    \item \textbf{Uczenie nienadzorowane}: Ten typ uczenia się dotyczy danych nieoznaczonych. Model próbuje identyfikować wzorce i relacje w danych bez żadnych wstępnie zdefiniowanych etykiet. Typowe zadania obejmują klasteryzację i redukcję wymiarowości.

    \item \textbf{Uczenie półnadzorowane}: To podejście łączy elementy uczenia nadzorowanego i nienadzorowanego. Model jest trenowany na małym zbiorze oznaczonych danych oraz dużym zbiorze danych nieoznaczonych. Umożliwia to wykorzystanie dostępnych danych w sposób bardziej efektywny, co jest szczególnie przydatne w sytuacjach, gdy oznaczanie danych jest kosztowne lub czasochłonne.
    

    \item \textbf{Uczenie się przez wzmacnianie}: W uczeniu się przez wzmacnianie agent uczy się podejmować decyzje, podejmując działania w środowisku w celu maksymalizacji skumulowanych nagród. Agent otrzymuje informacje zwrotne na podstawie swoich działań i odpowiednio dostosowuje swoją strategię.
\end{itemize}
% \clearpage

\phantomsection
\setstretch{1.5}
\section*{Dlaczego warto korzystać z uczenia maszynowego?}
\vspace{-1.0em}
\label{sec:why_use_ml}
\addcontentsline{toc}{section}{2. Dlaczego warto korzystać z uczenia maszynowego?}

\hspace{1.5cm} Uczenie maszynowe jest wykorzystywane w różnych domenach ze względu na jego zdolność do obsługi dużych ilości danych i odkrywania spostrzeżeń, które mogą nie być widoczne w przypadku tradycyjnych metod analitycznych. Kluczowe korzyści obejmują:

\begin{itemize}
    \item \textbf{Automatyzacja}: ML może automatyzować powtarzalne zadania, co prowadzi do zwiększenia wydajności.
    \item \textbf{Analiza predykcyjna}: Umożliwia organizacjom prognozowanie trendów i zachowań na podstawie danych historycznych.
    \item \textbf{Personalizacja}: Algorytmy ML mogą dostosowywać doświadczenia użytkowników, analizując ich preferencje i zachowania.
\end{itemize}

\phantomsection
\setstretch{1.5}
\section*{Główne wyzwania uczenia maszynowego}
\vspace{-1.0em}
\label{sec:ml_challenges}
\addcontentsline{toc}{section}{3. Główne wyzwania uczenia maszynowego}

\hspace{1.5cm} Chociaż uczenie maszynowe oferuje znaczące zalety, wiąże się również z kilkoma wyzwaniami:

\begin{itemize}
    \item \textbf{Jakość danych}: Wysokiej jakości, reprezentatywne dane są niezbędne do skutecznego szkolenia. 
    Dane niskiej jakości mogą prowadzić do niedokładnych modeli.
    \item \textbf{Nadmierne dopasowanie i niedostateczne dopasowanie}: Zrównoważenie złożoności modelu jest krytyczne. Nadmierne dopasowanie występuje, gdy model uczy się szumu z danych treningowych, podczas gdy niedostateczne dopasowanie występuje, gdy nie udaje mu się uchwycić podstawowych wzorców.
    \item \textbf{Dostrajanie hiperparametrów}: Wybór odpowiednich hiperparametrów może znacząco wpłynąć na wydajność modelu.
\end{itemize}
% \clearpage