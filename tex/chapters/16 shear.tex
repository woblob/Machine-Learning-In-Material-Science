\phantomsection
\setstretch{1.5}
\justify
\fontsize{14}{16}\selectfont
\setlength{\parindent}{0pt}
\section*{3. Moduł sprężystości poprzecznej \cite{alma991001031769708832}} 
\addcontentsline{toc}{section}{\textnormal{3. Moduł sprężystości poprzecznej}}
\fontsize{12}{14}\selectfont
% \vspace{\baselineskip} 

\hspace{1.5cm} Znany również jako moduł ścinania, określa reakcję materiału na naprężenie ścinające (równolegle do powierzchni).
Definiowany jest jako:

$$
G = \frac{\tau}{\gamma}
$$

gdzie:
\begin{itemize}
  \item $G$ to moduł ścinania,
  \item $\tau$ to naprężenie ścinające,
  \item $\gamma$ to odkształcenie ścinające.
\end{itemize}

\hspace{1.5cm} Siła ścinająca przyłożona do ciała stałego, powoduje że sąsiadujące warstwy materiału przesuwają się względem siebie. Moduł ścinania jest stałą określającą tę zależność między zastosowanym naprężeniem ścinającym a wynikającym z niego odkształceniem (odkształceniem ścinającym).

Materiały o wysokim module ścinania są w stanie wytrzymać znaczne odkształcenie bez uginania się, co czyni je idealnymi do zastosowań konstrukcyjnych.

% Chapter 1.2
\phantomsection
\setstretch{1.5}
\section*{Związek z innymi właściwościami mechanicznymi \cite{alma991000386409708832}}
\vspace{-1.0em}

\hspace{1.5cm} Moduł ścinania jest związany z innymi modułami sprężystości, takimi jak moduł Younga ($E$) lub wcześniej wspomniany, moduł objętościowy ($K$). W przypadku materiałów izotropowych związki te można wyrazić jako:

$$
G = \frac{E}{2(1 + \nu)} = \frac{3(1 \text{ - } 2\nu)K}{2(1 + \nu)}
$$

gdzie:
\begin{itemize}
  \item $\nu$ to współczynnik Poissona,
  \item $K$ to moduł sprężystości objętościowej,
  \item $E$ to moduł Younga
\end{itemize}


To równanie pokazuje, że moduł ścinania można przedstawiać w różny sposób, w zależności jakie właściwości materiału znamy. 

\clearpage

\hspace{1.5cm} W części praktycznej moduły sprężystości objętościowej i poprzecznej będą określane z metody uśrednień $"V R H"$ (skrót od nazwisk Voigt-Reuss-Hill) jako $K_{VRH}$ i $G_{VRH}$.

\hspace{1.5cm} Metody Voigt, Reuss i Hill \cite{alma991000386409708832, url_sparrow_przegl_papier_2024_05} są powszechnie stosowane w mechanice materiałów do określenia efektywnych modułów sprężystości w materiałach kompozytowych oraz wielofazowych.

\begin{itemize}

  \item {\textbf{Metoda Voigta} zakłada, że w całym materiale odkształcenie jest stałe. W wyniku, efektywny moduł sprężystości (np. Younga) oblicza się jako średnią arytmetyczną modułów poszczególnych faz:
$$
  E_V = \sum{E_iV_i}
$$
gdzie $E_i$ i $V_i$ to moduł Younga i udział procentowy dla fazy $i$.}

  \item {\textbf{Metoda Reussa} zakłada, że wszystkie fazy materiału są poddawane tym samym odkształceniom. W wyniku tego efektywny moduł sprężystości jest obliczany jako średnia harmoniczna:
  $$
  \frac{1}{E_R} = \sum\frac{V_i}{E_i}
  $$

gdzie $E_i$ i $V_i$ to moduł Younga i udział procentowy dla fazy $i$.
To daje dolną granicę dla efektywnego modułu sprężystości.}
  \item {\textbf{Metoda Hilla} łączy podejścia Voigta i Reussa, oferując uśredniony wynik pomiędzy tymi dwiema metodami. 

Efektywny moduł Younga $E_H$ oblicza się jako:
  $$
  E_H = \frac{1}{2}\left(E_V + E_R\right)
  $$
  co umożliwia uzyskanie wartości pośredniej między wartościami uzyskanymi z obu poprzednich metod.
}

\end{itemize}

\clearpage