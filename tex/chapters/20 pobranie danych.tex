\phantomsection
\setstretch{1.5}
% \justify
\fontsize{14}{16}\selectfont
\setlength{\parindent}{0pt}
\section*{2. Pobieranie danych z \textit{Materials Project}} 
\addcontentsline{toc}{section}{\textnormal{2. Pobieranie danych z \textit{Materials Project}}}
\fontsize{12}{14}\selectfont

\hspace{1.5cm} Na początku programu potrzeba danych. 
Ta funkcja zapewnia że dane zostaną zapisane na dysku. 
W przypadku późniejszego przerwania uczenia, zawsze można przywrócić zapisany stan.

Jeśli podano nazwę pliku, ta funkcja spróbuje załadować plik z dysku lokalnego o tej nazwie. 
Jeśli plik nie istnieje, to funkcja zacznie proces przygotowawczy bazy danych od podstaw.

\begin{pythoncode}
def get_df(filename=None) -> pd.DataFrame:
    if filename is None:
        filename = "outputs/df.joblib"

    df = None
    try:
        df = joblib.load(filename)
        print(f"Loaded full dataframe with {len(df)} records from {filename}")
    except FileNotFoundError as e:
        print(e)

        data = download_data_from_mp("outputs/mp_docs.joblib")
        df = get_full_dataframe(data)

        joblib.dump(df, filename)

    if df is None:
        raise Exception("df is still None")

    return df
\end{pythoncode}


Przed pobieraniem, trzeba zaciągnąć odpowiednie metody i zdefiniować klucz API do serwisu \textit{Materials Project}. 

\begin{pythoncode}
import os
from dotenv import load_dotenv
from mp_api.client import MPRester
from emmet.core.summary import HasProps
import pandas as pd
import joblib

from workFiles.functions.field_names import (
    fields_for_request,
    fields_to_drop,
    all_fields,
)

load_dotenv()
API_KEY = os.getenv("API_KEY")
modulus_limit = 1000
\end{pythoncode}

\hspace{1.5cm} Jeżeli obiekt z danymi istnieje, to on zostanie zwrócony. W innym przypadku następuje pobieranie przez API. 
Domyślnie ustawione są pola \textit{is\_stable}=True oraz \textit{theoretical}=False. Chcemy ściągnąć materiały termodynamicznie stabilne które zostały zbadane empirycznie. Przy ustawieniach domyślnych, dostajemy 5028 różnych materiałów jako odpowiedź API.

\hspace{1.5cm} W następnym kroku otrzymane dane są czyszczone przez funkcje pomocnicze.
Usuwane są nie potrzebne kolumny i przygotowywane są rekordy do późniejszej łatwiejszej obsługi.
Wyrzucane są nie pełne dane. 
Po czyszczeniu odjętych jest 1889, więc zostaje 3139 rekordów w bazie.  
Na koniec, wynik zapisujemy lokalnie i zwracamy.
\begin{pythoncode}
def download_data_from_mp(filename: str = None) -> pd.DataFrame:
    if filename is not None:
        try:
            df = joblib.load(filename)
            print(f"Loaded {len(df)} records from {filename}")
            return df
        except FileNotFoundError as e:
            print(e)

    with MPRester(API_KEY) as mpr:
        docs = mpr.materials.summary.search(
            theoretical=False,
            is_stable=True,
            has_props=[HasProps.elasticity],
            fields=fields_for_request,
        )

    print(f"Got {len(docs)} records")

    df = pd.DataFrame(docs, columns=all_fields)
    df.drop(fields_to_drop, axis=1, inplace=True)
    df = df.transform(_mp_data_unpack, axis=0)
    df.set_index("material_id", inplace=True)
    df.dropna(axis=0, how="any", inplace=True)
    df["bulk_modulus"] = df["bulk_modulus"].transform(_mp_data_modulus_cleaning)
    df["shear_modulus"] = df["shear_modulus"].transform(_mp_data_modulus_cleaning)
    df["universal_anisotropy"] = df["universal_anisotropy"].transform(
        _mp_data_anisotropy_cleaning
    )
    df.dropna(axis=0, how="any", inplace=True)

    joblib.dump(df, filename)

    print(f"Dropped {len(docs) - len(df)} records")

    return df
\end{pythoncode}

\needspace{4\baselineskip}
\hspace{1.5cm} Funkcje pomocnicze przy czyszczeniu. 
Pomimo że dane są oznaczone jako 'stabilne', to i tak niektóre wartości numeryczne bywają albo negatywne albo skrajnie duże.
\begin{pythoncode}
def _mp_data_unpack(ser: pd.Series) -> pd.Series:
    return ser.map(lambda x: x[1])


def _mp_data_modulus_cleaning(dictionary: dict) -> float | None:
    vrh_value = dictionary.get("vrh")

    if (vrh_value is not None) and 0 < vrh_value < modulus_limit:
        return vrh_value

    return None


def _mp_data_anisotropy_cleaning(value: float | None) -> float | None:
    if value is None or not (0 < value < 3):
        return None
    return value
\end{pythoncode}
