\phantomsection
\setstretch{1.5}
\justify
\fontsize{14}{16}\selectfont
\setlength{\parindent}{0pt}
\needspace{10\baselineskip}
\section*{6. Stworzenie wykresów} 
\addcontentsline{toc}{section}{\textnormal{6. Stworzenie wykresów}}
\fontsize{12}{14}\selectfont

\hspace{1.5cm} Ostatnim krokiem jest stworzenie wykresów. 
Dla każdego modelu względem cechy został stworzony wykres i umieszczony w następnym rozdziale.

\begin{pythoncode}
for feature in Results:
    for model_name in Results[feature]:
        instance = Results[feature][model_name]['instance']

        property_name = f'{featureMap[feature]} {unitsMap[feature]}'

        pf = PlotlyFig(x_title=f'Wartość rzeczywista:\n{property_name}',
                       y_title=f'Wartość przewidziana:\n{property_name}',
                       title=model_name,
                       filename=f"figures/{feature}-{model_name}.html",
                       fontsize=20)

        y = data.y[feature]
        X = data.X

        y_pred = instance.best_estimator_.predict(X)

        pf.xy(xy_pairs=[(y, y_pred), bounds[feature]],
              labels=df['formula_pretty'],
              modes=['markers', 'lines'],
              lines=[{}, {'color': 'black', 'dash': 'dash'}],
              showlegends=False
              )
\end{pythoncode}