\phantomsection
\setstretch{1.5}
\justify
\fontsize{14}{16}\selectfont
\setlength{\parindent}{0pt}
\section*{II. Uczenie maszynowe: Rodzaje \cite{alma991000280759708832}}
\addcontentsline{toc}{chapter}{II. Uczenie maszynowe: Rodzaje}
\fontsize{12}{14}\selectfont
\vspace{-1.0em}

\hspace{1.5cm} Uczenie maszynowe (ML) można ogólnie podzielić na dwa główne rodzaje: \textbf{klasyfikację} i \textbf{regresję}. Obie są podstawowymi technikami stosowanymi w uczeniu nadzorowanym, ale służą różnym celom i są stosowane w różnych kontekstach. Ten przegląd skupi się głównie na regresji, szczegółowo opisując jej cechy, zastosowania i metodologie.

\phantomsection
\setstretch{1.5}
\section*{Klasyfikacja kontra regresja}

\vspace{-1.0em}

\begin{itemize}
    \item \textbf{\textit{Klasyfikacja}}: To zadanie obejmuje przewidywanie wyników kategorycznych. Na przykład model klasyfikacji może przewidywać, czy wiadomość e-mail jest spamem, czy nie (\textit{klasyfikacja binarna}) lub klasyfikować obrazy do wielu kategorii (\textit{klasyfikacja wieloklasowa}). Wynik jest dyskretny, reprezentujący etykiety klas.
    \item \textit{\textbf{Regresja}}: Natomiast regresja dotyczy przewidywania ciągłych wartości liczbowych. Na przykład modele regresji mogą przewidywać ceny domów na podstawie różnych cech, takich jak rozmiar, lokalizacja i liczba sypialni. Wynik jest liczbą rzeczywistą, co czyni ją odpowiednią do zadań, których celem jest oszacowanie ilości.
\end{itemize}

\section*{Definicja regresji}

\vspace{-1.0em}

\hspace{1.5cm} Analiza regresji to metoda statystyczna stosowana do modelowania relacji między zmienną zależną (wynikiem) a jedną lub większą liczbą zmiennych niezależnych (predyktorami). Głównym celem regresji jest określenie, w jaki sposób zmiany zmiennych predykcyjnych wpływają na zmienną wynikową.

\section*{Powszechne zastosowania regresji}

\vspace{-1.0em}

\begin{itemize}
    \item \textbf{\textit{Finanse}}: przewidywanie cen akcji lub wskaźników ekonomicznych na podstawie danych historycznych.
    \item \textbf{\textit{Nieruchomości}}: prognozowanie wartości nieruchomości na podstawie lokalizacji, wielkości i trendów rynkowych.
    \item \textbf{\textit{Marketing}}: analiza zachowań konsumentów w celu przewidywania wyników sprzedaży na podstawie wydatków na reklamę.
\end{itemize}

\needspace{4\baselineskip}
\section*{Metryki oceny regresji \cite{alma991000280759708832}}

\vspace{-1.0em}

\hspace{1.5cm} Aby ocenić wydajność modeli regresji, powszechnie stosuje się kilka metryk:
\begin{itemize}
    \item \textbf{\textit{Średni błąd bezwzględny (MAE)}}: Mierzy średnią wielkość błędów wykorzystując tę samą skalę co mierzone dane. Nazywana też normą $L_{1}$.
    \[
    MAE = \frac{1}{n} \sum_{i=1}^{n} |y_i - \hat{y}_i|
    \]

    \item \textbf{\textit{Średni błąd kwadratowy (MSE)}}: Mierzy średnią różnicę kwadratową między przewidywanymi a rzeczywistymi wartościami. Większe błędy są przez niego karane bardziej niż MAE. Nazywana też normą $L_{2}$.
    \[
    MSE = \frac{1}{n} \sum_{i=1}^{n} (y_i - \hat{y}_i)^2
    \]

    \item \textbf{\textit{R-kwadrat ($R^2$)}}: Wskazuje, jak dobrze zmienne niezależne wyjaśniają zmienność zmiennej zależnej. Wartości mieszczą się w zakresie od 0 do 1, przy czym wyższe wartości oznaczają lepsze dopasowanie modelu.
\end{itemize}

% \clearpage
