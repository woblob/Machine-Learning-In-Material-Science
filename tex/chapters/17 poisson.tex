\phantomsection
\setstretch{1.5}
\justify
\fontsize{14}{16}\selectfont
\setlength{\parindent}{0pt}
\section*{4. Współczynnik Poisson'a \cite{alma991001031769708832}} 
\addcontentsline{toc}{section}{\textnormal{4. Współczynnik Poisson'a}}
\fontsize{12}{14}\selectfont
% \vspace{\baselineskip} 

\hspace{1.5cm} Określa on stosunek odkształcenia poprzecznego do odkształcenia osiowego w materiale poddanym rozciąganiu lub ściskaniu. Parametr ten pozwala zrozumieć, w jaki sposób substancja deformuje się w kierunkach prostopadłych do kierunku przyłożonego obciążenia.

Matematycznie jest to wyrażone jako:

$$
\nu = -\frac{\epsilon_n}{\epsilon_m}
$$

gdzie:
\begin{itemize}
    \item $\nu$ to współczynnik Poissona,
    \item $\epsilon_n$ to odkształcenie poprzeczne,
    \item $\epsilon_m$ to odkształcenie osiowe.
\end{itemize}

Typowe wartości współczynnika Poissona mieszczą się w zakresie od 0 do 0,5 dla większości materiałów, przy czym wartości bliskie 0,5 wskazują, że materiał zachowuje swoją objętość podczas odkształcania (materiał nieściśliwy).