\phantomsection
\setstretch{1.5}
\justify
\fontsize{14}{16}\selectfont
\setlength{\parindent}{0pt}
\section*{1. Regresja liniowa \cite{alma991000280759708832}}
\label{sec:machine_learning_overview}
\addcontentsline{toc}{section}{1. Regresja liniowa}
\fontsize{12}{14}\selectfont
\vspace{-1.0em}



Regresja liniowa to podstawowa metoda statystyczna stosowana w uczeniu maszynowym do modelowania relacji między jedną lub większą liczbą zmiennych niezależnych (predyktorów) a zmienną zależną (wynikiem). Głównym celem regresji liniowej jest znalezienie krzywej regresji, która najlepiej przewiduje zmienną zależną na podstawie zmiennych niezależnych.



\phantomsection
\setstretch{1.5}
\section*{Główne założenia \cite{url_linear_regression_assumptions, url_slowniki_regresja_liniowa}}
\vspace{-1.0em}
\label{sec:what_is_ml}
Regresja liniowa, jako jedna z podstawowych metod analizy statystycznej, opiera się na kilku kluczowych założeniach, które są niezbędne do prawidłowego funkcjonowania modelu. 

\begin{itemize}

    \item {Pierwszym z tych założeń jest \textbf{liniowość}, co oznacza, że związek między zmiennymi niezależnymi a zmienną zależną jest liniowy. Oznacza to, że zmiany w predyktorach powinny prowadzić do proporcjonalnych zmian w wyniku.}
   
    \item {Kolejnym istotnym założeniem jest \textbf{niezależność} obserwacji. Każda obserwacja powinna być niezależna od pozostałych, co jest kluczowe dla uzyskania wiarygodnych wyników.}
    
    \item {Trzecim założeniem jest \textbf{homoskedastyczność}, które odnosi się do stałej wariancji reszt, czyli błędów modelu, na wszystkich poziomach zmiennych niezależnych. To zapewnia, że nie występują systematyczne różnice w błędach w różnych zakresach wartości zmiennych.}
    
    \item {Ostatnim jest \textbf{normalizacja} reszt i błędów. Oczekuje się, że błędy modelu będą miały rozkład normalny, co jest istotne dla przeprowadzania testów statystycznych i oceny jakości modelu.}
    
\end{itemize}


\phantomsection
\setstretch{1.5}
\section*{Wzór ogólny}
\vspace{-1.0em}
\label{sec:what_is_ml}


Dla prostego modelu regresji liniowej z jednym predyktorem to:

$$
\hat y = \beta_0 1 + \beta_1 x + \epsilon
$$

gdzie: \\
- $\hat y$ to przewidywana wartość (zmienna zależna). \\
- $x$ to zmienna niezależna (predyktor). \\
- $\beta_0$ to przecięcie y linii regresji. \\
- $\beta_1$ to nachylenie linii regresji, reprezentujące zmianę $y$ dla zmiany $x$ o jedną jednostkę. \\ 
- $\epsilon$ jest błędem statystycznym, uwzględniającym zmienność w $y$ niewyjaśnioną przez $x$ . \\

W przypadku wielu zmiennych regresji liniowej, w których występuje wiele predyktorów, wzór rozszerza się do:

$$
y_i = \beta_0 + \beta_1 x_1 + \beta_2 x_2 + \ldots + \beta_n x_n + \epsilon_i = \boldsymbol{x}_i^{\top} \boldsymbol{\beta} + \epsilon_i
$$

gdzie: \\
- $x_1, x_2, \ldots, x_n$ są zmiennymi niezależnymi. \\
- $\beta_1, \beta_2, \ldots, \beta_n$ są współczynnikami odpowiadającymi każdemu predyktorowi. \\

Najogólniejszy zapis macierzowy:
$$
\mathbf{y} = X {\boldsymbol{\beta}} + {\boldsymbol{\epsilon}}
$$

gdzie:
$$
\mathbf{y} = { \begin{pmatrix} y_{1} \\ y_{2} \\ \vdots \\y_{n} \end{pmatrix} }, \qquad X = {\begin{pmatrix} \mathbf{x}_{1}^{\top} \\ \mathbf{x}_{2}^{\top} \\ \vdots \\ \mathbf{x}_{n}^{\top} \end{pmatrix} } = { \begin{pmatrix} 1 & x_{11} & \ldots & x_{1p} \\ 1 & x_{21} & \ldots & x_{2p} \\ \vdots & \vdots & \ddots & \vdots \\ 1 & x_{n1} & \ldots & x_{np} \end{pmatrix}},\\ \qquad{ \boldsymbol{\beta}} = {\begin{pmatrix} \beta_{0} \\ \beta_{1} \\ \vdots \\ \beta_{p} \end{pmatrix}}, \qquad{ \boldsymbol{\epsilon}} = {\begin{pmatrix} \epsilon_{1} \\ \epsilon_{2} \\ \vdots \\ \epsilon_{n} \end{pmatrix}}
$$
\\

\phantomsection
\setstretch{1.5}
\section*{Użyte parametry dla GridSearchCV \cite{url_LinearRegression, url_grid_search}}
\vspace{-1.0em}
% \label{sec:why_use_ml}

\begin{itemize}
    \item positive: określa czy współczynniki regresji powinny być dodatnie. 

 \item  fit\_intercept: parametr logiczny wskazuje, czy obliczyć przecięcie dla tego modelu. Krytyczne dla danych standaryzowanych. 
\end{itemize}
 
% \noindent\makebox[\linewidth]{\rule{\paperwidth}{0.4pt}}
\clearpage