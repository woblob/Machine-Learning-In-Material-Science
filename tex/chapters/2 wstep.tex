

% Chapter 1
\phantomsection
\setstretch{1.5}
% \justify
\fontsize{14}{16}\selectfont
\setlength{\parindent}{0pt}
% \setcounter{chapter}{1}
\chapter*{WSTĘP} 
% \pagenumbering{arabic} 
\addcontentsline{toc}{chapter}{\textnormal{Wstęp}}
\fontsize{12}{14}\selectfont
\vspace{\baselineskip} 
% Content goes here

% Chapter 1.1
\phantomsection
\setstretch{1.5}
% \section*{Wstęp teoretyczny} 
\vspace{-1.0em}
Zastosowanie uczenia maszynowego w inżynierii materiałowej stało się podejściem, umożliwiającym badaczom i inżynierom przewidywanie właściwości materiałów, optymalizację procesów i bardziej wydajne wprowadzanie innowacji w zakresie nowych materiałów. 





% Chapter 1.2
\phantomsection
\setstretch{1.5}
\section*{1. Przegląd uczenia maszynowego}
\vspace{-1.0em}
\addcontentsline{toc}{section}{1. Przegląd uczenia maszynowego}
Uczenie maszynowe jest rodzajem sztucznej inteligencji. 
Koncentruje się na opracowywaniu algorytmów, które pozwalają komputerom uczyć się i formułować przewidywania na podstawie danych. 
W przeciwieństwie do tradycyjnego programowania, w którym dostarczane są wyraźne instrukcje, systemy ML poprawiają swoją wydajność, gdy są narażone na więcej danych. 
Ta możliwość jest szczególnie cenna w inżynierii materiałowej, w której często występują złożone relacje między zmiennymi.

\begin{itemize}

\item {\textbf{\textit{Typy uczenia maszynowego}}}

Dwiema głównymi kategoriami uczenia maszynowego są uczenie nadzorowane i nienadzorowane.
Uczenie nadzorowane polega na trenowaniu modelu na danych oznaczonych, podczas gdy uczenie nienadzorowane zajmuje się danymi bez wstępnie zdefiniowanych etykiet.
W tej sekcji zostaną omówione różne paradygmaty uczenia maszynowego istotne dla przewidywania i klasyfikacji właściwości materiałów.




\item {\textbf{\textit{Znaczenie w inżynierii materiałowej}}

Integracja technik uczenia maszynowego umożliwia ulepszoną analizę zachowania materiałów w różnych warunkach, co prowadzi do ulepszonych procesów projektowania i selekcji.
Wykorzystując dane historyczne, uczenie maszynowe może identyfikować wzorce, które informują o rozwoju materiałów.
}



\end{itemize}
% \clearpage


% Chapter 1.3
\phantomsection
\setstretch{1.5}
\needspace{4\baselineskip}
\section*{2. Wyzwania w uczeniu maszynowym}
\vspace{-1.0em}

Pomimo swojego potencjału, kilka wyzwań utrudnia skuteczne zastosowanie uczenia maszynowego w inżynierii materiałowej:
\begin{enumerate}

\needspace{4\baselineskip}
\item {\textbf{\textit{Jakość danych}}}


Integracja technik uczenia maszynowego umożliwia ulepszoną analizę zachowania materiałów w różnych warunkach, co prowadzi do ulepszonych procesów projektowania i selekcji.
Wykorzystując dane historyczne, uczenie maszynowe może identyfikować wzorce, które informują o rozwoju materiałów.

\item {\textbf{\textit{Wybór cech}}}


Identyfikacja istotnych cech, które znacząco wpływają na wydajność modelu, ma kluczowe znaczenie.
Nieistotne lub zbędne cechy mogą obniżyć dokładność modelu.

\item {\textbf{\textit{Dostrajanie hiperparametrów}}}


Wybór odpowiednich hiperparametrów ma zasadnicze znaczenie dla optymalizacji wydajności modelu.
W tej sekcji zostaną omówione różne strategie skutecznego dostrajania hiperparametrów.
\end{enumerate}

% Chapter 1.3
\phantomsection
\setstretch{1.5}
\section*{3. Metodologia uczenia maszynowego w inżynierii materiałowej}
\vspace{-1.0em}
\addcontentsline{toc}{section}{3. Metodologia uczenia maszynowego w inżynierii materiałowej}
Ta praca przedstawi metodologię wdrażania uczenia maszynowego w inżynierii materiałowej:


\begin{enumerate}
    \item \textbf{\textit{Pozyskiwanie danych}}


Integracja technik uczenia maszynowego umożliwia ulepszoną analizę zachowania materiałów w różnych warunkach, co prowadzi do ulepszonych procesów projektowania i selekcji.
Wykorzystując dane historyczne, uczenie maszynowe może identyfikować wzorce, które informują o rozwoju materiałów.

\item \textbf{\textit{Inżynieria cech}}


Zostanie przedstawiony szczegółowy opis procesu ekstrakcji cech, podkreślający, w jaki sposób surowe dane są przekształcane w znaczące dane wejściowe dla modeli ML.


\item \textbf{\textit{Rozwój modelu}}


Zbadane zostaną różne modele regresji, w tym regresja liniowa, maszyny wektorów nośnych (SVM), drzewa decyzyjne i metody zespołowe.
Oceniona zostanie stosowalność każdego modelu do konkretnych problemów w inżynierii materiałowej.
\end{enumerate}

\phantomsection
\setstretch{1.5}

\section*{4. Analiza wyników}
\vspace{-1.0em}
\addcontentsline{toc}{section}{4. Analiza wyników}
W tej sekcji zaprezentowane zostaną osiągi wdrożonych modeli. Przeprowadzona zostanie analiza porównawcza wszystkich zastosowanych modeli, mająca na celu ocenę ich dokładności predykcyjnej w odniesieniu do danych rzeczywistych. Wydajność poszczególnych modeli zostanie zilustrowana za pomocą reprezentacji graficznych, takich jak wykresy punktowe.



\phantomsection
\setstretch{1.5}
\section*{5. Wnioski}
\vspace{-1.0em}
\addcontentsline{toc}{section}{5. Wnioski}
W tej sekcji podsumowane zostaną przeprowadzone badania, ze szczególnym uwzględnieniem modeli uczenia maszynowego, które wykazały się największą skutecznością w przewidywaniu właściwości materiałów. Dodatkowo omówione zostaną implikacje tych wyników dla przyszłych badań oraz praktycznych zastosowań w inżynierii materiałowej.
\clearpage