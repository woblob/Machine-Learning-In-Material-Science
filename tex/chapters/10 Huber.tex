\phantomsection
\setstretch{1.5}
\justify
\fontsize{14}{16}\selectfont
\setlength{\parindent}{0pt}
\section*{6. Regresor Hubera \cite{huber1964robust}}
\addcontentsline{toc}{section}{6. Regresor Hubera}
\fontsize{12}{14}\selectfont
\vspace{-1.0em}

\hspace{1.5cm} Regresor Hubera łączy zasady najmniejszych kwadratów i funkcje kosztu błędu bezwzględnego. Jest skuteczna w scenariuszach, w których zbiór danych zawiera wartości odstające, ponieważ zmniejsza ich wpływ na przewidywania modelu. 

\phantomsection
\setstretch{1.5}
\section*{Zasada działania \cite{hastie2009elements}}
\vspace{-1.0em}

Funkcja kosztu Hubera jest zdefiniowana następująco:

$$
L_\delta( y, \hat{y} ) = \begin{cases}
    ( y - \hat{y} )^2 & \text{for} | y - \hat{y} | \leq \delta \\
    2 \delta  | y - \hat{y} | - \delta^2  & \text{for} | y - \hat{y} | > \delta
\end{cases}
$$

\needspace{4\baselineskip}
gdzie:
\begin{itemize}
\setlength\itemsep{-0.5em}
\item $y$ jest wartością rzeczywistą.

 \item  $\hat{y}$ jest wartością przewidywaną.

 \item  $\delta$ jest parametrem progowym, który określa punkt, w którym funkcja kosztu przechodzi z kwadratowej na liniową.
\end{itemize}

\hspace{1.5cm} Funkcja kosztu Hubera jest zdefiniowana w taki sposób, że dla reszt (różnic między wartościami rzeczywistymi a przewidywanymi) mniejszych niż parametr progowy $\delta$, koszt zachowuje się jak błąd kwadratowy. To oznacza, że jest bardziej wrażliwy na małe błędy. Natomiast dla reszt większych niż $\delta$ koszt przechodzi na liniową formę. Sprawia to, że model staje się mniej wrażliwy na wartości odstające. 

Wybór parametru $\delta$ w regresorze Hubera jest kluczowy dla wydajności modelu. Mniejsza wartość $\delta$ sprawia, że model staje się bardziej wrażliwy na wartości odstające, co może być korzystne w przypadku danych o niskiej liczbie anomalii, natomiast większa wartość $\delta$ zwiększa odporność modelu na wartości odstające, co czyni go bardziej stabilnym w trudniejszych warunkach.
% \clearpage

\phantomsection
\setstretch{1.5}
\section*{Zalety }
\vspace{-1.0em}

\hspace{1.5cm} Regresor Hubera cechuje się małą wrażliwością na wartości odstające w porównaniu ze zwykłą regresją najmniejszych kwadratów. Dzięki temu jest preferowanym wyborem w przypadku zbiorów danych z obserwacjami anomalnymi. Ponadto, oferuje płynne przejście między kwadratowymi i liniowymi funkcjami kosztów, co zapewnia równowagę między wrażliwością na małe błędy a odpornością na duże błędy.


\phantomsection
\setstretch{1.5}
\section*{Ograniczenia }
\vspace{-1.0em}
\label{sec:ml_challenges}

\hspace{1.5cm} Wybór parametru $\delta$ może znacząco wpłynąć na wydajność modelu. Wybór odpowiedniej wartości może wymagać walidacji krzyżowej lub wiedzy o domenie.
Ponadto, regresja Hubera może być bardziej intensywna obliczeniowo niż prostsze modele, szczególnie w przestrzeniach wielowymiarowych. Model ten może nie być tak skuteczny w przypadkach, gdy dane są silnie skorelowane lub gdy istnieją silne nieliniowe relacje między zmiennymi.


\phantomsection
\setstretch{1.5}
\section*{Zastosowania}
\vspace{-1.0em}
\label{sec:ml_challenges}

\hspace{1.5cm} Ten model jest często preferowanym wyborem w przypadkach, gdy dane mogą zawierać anomalie.
Regresor Hubera zapewnia solidną alternatywę dla tradycyjnych technik regresji.
Łączy zalety zarówno metod najmniejszych kwadratów, jak i metod błędów bezwzględnych.
Dzięki temu nadaje się do zestawów danych z wartościami odstającymi, przy jednoczesnym zachowaniu dobrej wydajności predykcyjnej.


\phantomsection
\setstretch{1.5}
\needspace{4\baselineskip}
\section*{Użyte parametry dla GridSearchCV \cite{url_HuberRegressor, url_grid_search}}
\vspace{-1.0em}
\label{sec:ml_challenges}

\begin{itemize}
\setlength\itemsep{-0.5em}
\item max\_iter: określa maksymalną liczbę iteracji algorytmu optymalizacji.
\item epsilon: określa próg klasyfikacji reszt jako wartości odstających. Za niska wartość może prowadzić do niedopasowania, natomiast za wysoka, do przetrenowania.
\item alpha: kontroluje siłę regularyzacji stosowaną do modelu.
\item fit\_intercept: parametr logiczny wskazuje, czy obliczyć przecięcie dla tego modelu. Krytyczne dla danych standaryzowanych. 
\end{itemize}
% \clearpage
% \noindent\makebox[\linewidth]{\rule{\paperwidth}{0.4pt}}

