\phantomsection
\setstretch{1.5}
\justify
\fontsize{14}{16}\selectfont
\setlength{\parindent}{0pt}
\chapter*{III. Właściwości mechaniczne \cite{alma991000386409708832, kittel1976wstep, alma991001031769708832, alma991053253943408832}} 
\addcontentsline{toc}{chapter}{\textnormal{III. Właściwości mechaniczne}}
\fontsize{12}{14}\selectfont
\vspace{\baselineskip} 

\hspace{1.5cm} Aby zrozumieć, jak materiały reagują na przyłożone siły, należy zacząć od prawa Hooke'a, które jest fundamentalne w badaniu właściwości mechanicznych ciał stałych. \\
\hspace{1.5cm} \textbf{Prawo Hooke'a} stwierdza, że odkształcenie (deformacja) materiału jest wprost proporcjonalne do przyłożonego naprężenia (siła na jednostkę powierzchni) w granicach sprężystości tego materiału. Matematycznie, uproszczoną wersję prawa Hooke'a można wyrazić jako \cite{alma991053253943408832, alma991000386409708832}:

$$
\sigma = E \cdot \epsilon
$$

gdzie:
\begin{itemize}
 \item  $\sigma$ to naprężenie,
 \item  $E$ to moduł sprężystości (moduł Younga),
 \item  $\epsilon$ to odkształcenie sprężyste.
\end{itemize}


Pełną ogólną postać macierzową, która wiąże naprężenie i odkształcenie w ciałach stałych, można wyrazić za pomocą macierzy $6 \times 6$. Ta formuła uwzględnia złożoność materiałów anizotropowych, w których właściwości mechaniczne zmieniają się w zależności od kierunku.
% Chapter 1.2
\phantomsection
\setstretch{1.5}
\section*{1. Uogólnione prawo Hooke'a w postaci macierzowej \cite{kittel1976wstep, alma991001031769708832}}
\vspace{-1.0em}
\label{sec:spacing_font_size}
\addcontentsline{toc}{section}{1. Uogólnione prawo Hooke'a w postaci macierzowej}
\hspace{1.5cm} W przypadku materiału liniowo sprężystego zależność między naprężeniem $\sigma$ a odkształceniem $\epsilon$ można przedstawić jako:

$$
\begin{bmatrix}
\sigma_1 \\
\sigma_2 \\
\sigma_3 \\
\tau_{23} \\
\tau_{13} \\
\tau_{12}
\end{bmatrix}
=
\begin{bmatrix}
C_{11} & C_{12} & C_{13} & C_{14} & C_{15} & C_{16} \\
C_{12} & C_{22} & C_{23} & C_{24} & C_{25} & C_{26} \\
C_{13} & C_{23} & C_{33} & C_{34} & C_{35} & C_{36} \\
C_{14} & C_{24} & C_{34} & C_{44} & C_{45} & C_{46} \\
C_{15} & C_{25} & C_{35} & C_{45} & C_{55} & C_{56} \\
C_{16} & C_{26} & C_{36} & C_{46} & C_{56} & C_{66}
\end{bmatrix}
\begin{bmatrix}
\epsilon_1 \\
\epsilon_2 \\
\epsilon_3 \\
\gamma_{23} \\
\gamma_{13} \\
\gamma_{12}
\end{bmatrix}
$$

gdzie $\sigma$ odpowiada wektorowi naprężeń, $E$ macierzy sztywności sprężystej, a $\epsilon$ wektorowi odkształceń.

\hspace{1.5cm}  \textbf{Wektor naprężeń}: obejmuje naprężenia normalne ($\sigma_1, \sigma_2, \sigma_3$) i naprężenia ścinające ($\tau_{23}, \tau_{13}, \tau_{12}$).

\hspace{1.5cm}  \textbf{Macierz sztywności sprężystej}: Macierz $C$ (lub macierz sztywności) zawiera stałe materiałowe, które definiują, w jaki sposób naprężenie odnosi się do odkształcenia. Każdy $C_{ij}$ reprezentuje związek między składową $i$ naprężenia a składową $j$ odkształcenia.

\hspace{1.5cm}  \textbf{Wektor odkształceń}: składa się z odkształceń normalnych ($\epsilon_1, \epsilon_2, \epsilon_3$) i odkształceń ścinających ($\gamma_{23}, \gamma_{13}, \gamma_{12}$).

\hspace{1.5cm} Ze względu na symetryczną naturę zarówno tensorów naprężeń, jak i odkształceń w sprężystości liniowej, macierz sztywności jest również symetryczna. Oznacza to, że $C_{ij}=C_{ji}$, co zmniejsza liczbę niezależnych stałych potrzebnych do pełnego opisu zachowania materiału.

\hspace{1.5cm} Ta uogólniona forma jest kluczowa dla analizy złożonych warunków obciążenia w zastosowaniach inżynieryjnych, szczególnie w materiałach, które nie wykazują jednolitych właściwości we wszystkich kierunkach (materiały \textit{anizotropowe}). Zrozumienie tej zależności pozwala przewidywać, jak materiały będą się zachowywać pod różnymi typami obciążeń.

\hspace{1.5cm} Uogólnioną formę prawa Hooke'a można zredukować, uwzględniając symetrie materiału. 

\hspace{1.5cm} Na przykład w przypadku materiałów izotropowych, które mają jednorodne właściwości we wszystkich kierunkach, liczba niezależnych współczynników w macierzy sztywności może być zmniejszona do nawet tylko dwóch. 

\hspace{1.5cm} W takich przypadkach można zastosować uproszczone modele, które uwzględniają tylko kilka kluczowych parametrów materiałowych, co upraszcza analizy i obliczenia związane z zachowaniem materiału pod wpływem obciążeń.
\needspace{4\baselineskip}

Symetryczna macierz dla materiału izotropowego:
$$
\begin{bmatrix}
C_{11} & C_{12} & C_{12} & 0 & 0 & 0 \\
. & C_{11} & C_{12} & 0 & 0 & 0 \\
. & . & C_{11} & 0 & 0 & 0 \\
. & . & . & \frac{C_{11} - C_{12}}{2} & 0 & 0 \\
. & . & . & . & \frac{C_{11} - C_{12}}{2} & 0 \\
. & . & . & . & . & \frac{C_{11} - C_{12}}{2}
\end{bmatrix}
$$

% \clearpage
 Ważne pojęcia \cite{kittel1976wstep, alma991053253943408832}:
\begin{itemize}
    \item {\textbf{Odkształcenie sprężyste} występuje, gdy materiały są poddawane naprężeniom, które nie są zbyt duże, co pozwala im powrócić do pierwotnego kształtu po usunięciu naprężenia. Tego rodzaju deformacja jest istotna w kontekście projektowania i analizy materiałów, ponieważ umożliwia im funkcjonowanie w warunkach zmiennych obciążeń bez ryzyka trwałych uszkodzeń.}
    
    \item {Prawo Hooke'a dotyczy zarówno rozciągania, jak i ściskania. Gdy ciało stałe jest wydłużane, jego wymiary zmniejszają się w kierunkach prostopadłych z powodu \textbf{efektu Poissona}. To zachowanie można zwizualizować, rozważając pręt cylindryczny: w miarę rozciągania staje się cieńszy w średnicy.}

    \item {\textbf{Moduł sprężystości} $E$ określa sztywność materiału i jest definiowany jako stosunek naprężenia do odkształcenia. Metale zazwyczaj posiadają wysokie wartości modułu sprężystości. Zazwyczaj są sztywne i odporne na odkształcenia pod obciążeniem, podczas gdy polimery mają niski moduł sprężystości, co czyni je bardziej elastycznymi.}

    \item {Zależność opisana przez prawo Hooke'a jest prawdziwa tylko do pewnej granicy znanej jako \textbf{granica sprężystości}. Po przekroczeniu tego punktu materiały mogą ulegać odkształceniom plastycznym, w wyniku czego nie powracają do swojego pierwotnego kształtu po usunięciu naprężenia.}
    
\end{itemize}


% \clearpage


\noindent\makebox[\linewidth]{\rule{\paperwidth}{0.4pt}}