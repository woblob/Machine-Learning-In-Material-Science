\phantomsection
\setstretch{1.5}
\justify
\fontsize{14}{16}\selectfont
\setlength{\parindent}{0pt}
\section*{3. Regresja grzbietowa \cite{alma991000280759708832}}
\addcontentsline{toc}{section}{3. Regresja grzbietowa}
\fontsize{12}{14}\selectfont
\vspace{-1.0em}

\hspace{1.5cm} Regresja grzbietowa to technika regresji liniowej, która wykorzystuje regularyzację $L_2$ w celu rozwiązania problemów współliniowości i nadmiernego dopasowania. Dodając człon kary do funkcji kosztu, regresja grzbietowa pomaga ustabilizować oszacowania współczynników, szczególnie gdy istnieją wysokie korelacje między predyktorami.

\phantomsection
\setstretch{1.5}
\needspace{4\baselineskip}
\section*{Wzóry ogólne}
\vspace{-1.0em}


\textbf{ Jawny wzór regresji grzbietowej}: 
\begin{center}
    $\boldsymbol{\hat \beta} = ( \boldsymbol{X}^T \boldsymbol{X} + \lambda \boldsymbol{I} )^{-1} \boldsymbol{X}^T \boldsymbol{y}$
\end{center}

gdzie:
\begin{itemize}
\setlength\itemsep{-0.5em}
 \item  $\boldsymbol{\hat \beta}$ to wektor szacowanych współczynników (lub wag) dla modelu regresji grzbietowej.
 \item  $\boldsymbol{X}$ to macierz cech zawierająca zmienne niezależne (predyktory) używane w regresji. Każdy wiersz odpowiada obserwacji, a każda kolumna odpowiada cesze.
\item $\boldsymbol{X}^T$ to transponowana macierz $\boldsymbol{X}$.
\item $\lambda$ to stopień regularyzacji. Kontroluje on ilość skurczu stosowanego do współczynników. Wyższa wartość $\lambda$ skutkuje większym skurczem, co pomaga zmniejszyć nadmierne dopasowanie poprzez karanie dużych wartości współczynników.
\item  $\boldsymbol{I}$: To przedstawia macierz jednostkową, która jest macierzą \\kwadratową z jedynkami na przekątnej i zerami w innych miejscach. 
\item  $\boldsymbol{y}$: To wektor odpowiedzi zawierający obserwacje zmiennej zależnej, którą próbujemy przewidzieć.

\end{itemize}
% \clearpage

    \textbf{Funkcja kosztu dla regresji grzbietowej}: 
    \begin{center}
        $L = \sum_{i=1}^{n} (y_i - (\boldsymbol{x}_i^{\top} \boldsymbol{\beta} + \epsilon_i))^2 + \lambda \sum_{j=1}^{n} \beta_j^2$
    \end{center}

gdzie:
\begin{itemize}
\setlength\itemsep{-0.5em}
 \item  $L$ jest całkowitą funkcją kosztu.
 \item  $y_i$ są rzeczywistymi wartościami zmiennej zależnej.

 \item  $\boldsymbol{x}_i^{\top}$ to transponowany wektor zmiennych niezależnych (cechy).

 \item  $\boldsymbol{\beta}$ to wektor współczynników modelu.

 \item  $\lambda$ to parametr regularyzacji, który kontroluje siłę kary stosowanej do współczynników.
\end{itemize}
\item 

\phantomsection
\setstretch{1.5}
\section*{Główne cechy}
\vspace{-1.0em}


\begin{itemize}
\item Regresja grzbietowa, wprowadza człon regularyzacyjny, który wymusza na modelu utrzymanie jak najmniejszych wartości wag. Ten człon, opisany jako $\lambda \sum_{j=1}^{n} \beta_j^2$, przeciwdziała nadmiernemu dopasowaniu poprzez zmniejszenie współczynników w kierunku zera, ale nie ustawia ich dokładnie na zero. Dzięki temu wszystkie cechy pozostają w modelu.

\item Regresja grzbietowa jest szczególnie skuteczna w sytuacjach, gdy zmienne predykcyjne są silnie skorelowane. Podobnie jak w przypadku modelu LASSO, regularyzacja zmniejsza wariancję i zapewnia skuteczniejsze oszacowania.
\end{itemize}

\phantomsection
\setstretch{1.5}
\section*{Zalety}
\vspace{-1.0em}


\begin{itemize}
\item Jedną z głównych zalet regresji grzbietowej jest stabilność oszacowań. W porównaniu do zwykłej regresji najmniejszych kwadratów, regresja grzbietowa dostarcza bardziej stabilnych wyników, szczególnie w przestrzeniach wielowymiarowych, gdzie występuje współliniowość między zmiennymi. 

\item Dodatkowo, metoda ta skutecznie zarządza kompromisem między odchyleniem a wariancją. Wprowadzając karę za rozmiar współczynników, regresja grzbietowa pomaga osiągnąć lepszą generalizację na niewidzianych danych, co jest istotne dla praktycznych zastosowań modelowania.
\end{itemize}

\phantomsection
\setstretch{1.5}
\section*{Ograniczenia}
\vspace{-1.0em}


\begin{itemize}
\item W przeciwieństwie do regresji Lasso, nie wykonuje wyboru zmiennej; wszystkie predyktory pozostają w modelu. To może być nieidealne z punktu widzenia łatwości interpretacji, zwłaszcza w przypadkach z wieloma cechami.

\item Kolejnym ograniczeniem jest wrażliwość na wybór parametru regularyzacji $\lambda$. Wartość zbyt wysoka może prowadzić do niedopasowania modelu do danych, podczas gdy zbyt niska wartość może nie wystarczająco rozwiązać problem nadmiernego dopasowania.
\end{itemize}

\phantomsection
\setstretch{1.5}
\needspace{4\baselineskip}
\section*{Użyte parametry dla GridSearchCV \cite{url_Ridge, url_grid_search}}
\vspace{-1.0em}

\begin{itemize}
\item {alpha}: siła regularyzacji. Kontroluje, jak bardzo karać duże współczynniki.

\item {solver}: określa algorytm używany do optymalizacji.
\end{itemize}
\noindent\makebox[\linewidth]{\rule{\paperwidth}{0.4pt}}
