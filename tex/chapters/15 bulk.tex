\phantomsection
\setstretch{1.5}
\justify
\fontsize{14}{16}\selectfont
\setlength{\parindent}{0pt}
\section*{2. Moduł sprężystości objętościowej \cite{alma991000389799708832, hyperphysics_url}} 
\addcontentsline{toc}{section}{\textnormal{2. Moduł sprężystości objętościowej}}
\fontsize{12}{14}\selectfont
% \vspace{\baselineskip} 

\hspace{1.5cm} Ten moduł mierzy odporność materiału na równomierne ściskanie. 
Określa, jak nieściśliwy jest materiał pod ciśnieniem. 
Matematycznie jest definiowany jako stosunek zmiany ciśnienia do zmiany objętości:

$$
K = -V \frac{dP}{dV}
$$

\needspace{4\baselineskip}
gdzie:
\begin{itemize}
    \item  $K$ - moduł sprężystości objętościowej,
    \item  $V$ - objętość,
    \item  $P$ - ciśnienie,
\end{itemize}

\hspace{1.5cm} Moduł sprężystości objętościowej wskazuje, jak nieściśliwy jest materiał. Im wyższy moduł sprężystości objętościowej, tym materiał jest mniej ściśliwy, co czyni go lepszym do zastosowań wymagających dużej odporności na zmiany objętości.

\hspace{1.5cm} Moduł objętościowy w ciele stałym wpływa na prędkość dźwięku i inne fale mechaniczne w materiale. Jest również czynnikiem wpływającym na ilość energii zgromadzonej w skorupie ziemskiej. Ta akumulacja energii sprężystej może zostać gwałtownie uwolniona podczas trzęsienia ziemi, więc znajomość modułów objętościowych dla materiałów skorupy ziemskiej jest ważną częścią badania trzęsień ziemi. 
% Chapter 1.2
\phantomsection
\setstretch{1.5}
\section*{Interpretacja fizyczna \cite{alma991000386409708832}}
\vspace{-1.0em}
\label{sec:spacing_font_size}
% \addcontentsline{toc}{section}{1.Interpretacja fizyczna macierzowej}
\hspace{1.5cm} Gdy do ciała stałego zostanie przyłożone ciśnienie zewnętrzne, powoduje to zmniejszenie objętości. Moduł sprężystości objętościowej stanowi ilościową miarę tego, jak bardzo objętość zmniejsza się w odpowiedzi na przyłożone ciśnienie. Dokładniej, opisuje, jakie ciśnienie należy przyłożyć, aby uzyskać określoną zmianę objętości.

\hspace{1.5cm} Moduł objętościowy może być związany z innymi właściwościami mechanicznymi materiałów, takimi jak moduł Younga ($E$) i współczynnik Poissona ($\nu$). W przypadku materiałów izotropowych, zależności te można wyrazić jako:

$$
K = \frac{E}{3(1 - 2\nu)}
$$

\hspace{1.5cm} To równanie pokazuje, że moduł objętościowy zależy zarówno od sztywności materiału (przedstawionej przez moduł Younga), jak i jego zdolności do odkształcania się na boki (przedstawionej przez współczynnik Poissona).

\clearpage