
\phantomsection
\setstretch{1.5}
\justify
\fontsize{14}{16}\selectfont
\setlength{\parindent}{0pt}
\section*{V. Wyniki } 
\addcontentsline{toc}{chapter}{\textnormal{V. Wyniki }}
\fontsize{12}{14}\selectfont
% \vspace{\baselineskip} 

\hspace{1.5cm} Powstały kod, można uruchomić w interpreterze języka Python.
Domyślne środowisko to "\textit{Jupyter Notebook}" ze względu na występowanie plików w formacie 'ipynb', ale popularne edytory tekstu jak naprzykład "VS Code", albo IDE "\textit{PyCharm}" też potrafią obsługiwać tego typu pliki.

\hspace{1.5cm} Zgrupowano wykresy względem modeli. Każdy wykres przedstawia efektywność modelu względem wartości rzeczywistych.
Na osi rzędnych są wartości przewidziane, na odciętych prawdziwe.


\begin{figure}[ht]
\centering
\subfigure[Moduł sprężystości objętościowej]{%
    \includegraphics[width=0.48\textwidth]{images/figures/newplot (0).png}
}
\subfigure[Moduł sprężystości poprzecznej]{%
    \includegraphics[width=0.48\textwidth]{images/figures/newplot (9).png}
}
\\
\subfigure[Współczynninik Anizotropii]{%
    \includegraphics[width=0.48\textwidth]{images/figures/newplot (18).png}
}
\subfigure[Liczba Poisona]{%
    \includegraphics[width=0.48\textwidth]{images/figures/newplot (27).png}
}
\caption{Regresja Liniowa}
\end{figure}


\clearpage
\begin{figure}[ht]
\centering
\subfigure[Moduł sprężystości objętościowej]{%
    \includegraphics[width=0.48\textwidth]{images/figures/newplot (1).png}
}
\subfigure[Moduł sprężystości poprzecznej]{%
    \includegraphics[width=0.48\textwidth]{images/figures/newplot (10).png}
}
\\
\subfigure[Współczynninik Anizotropii]{%
    \includegraphics[width=0.48\textwidth]{images/figures/newplot (19).png}
}
\subfigure[Liczba Poisona]{%
    \includegraphics[width=0.48\textwidth]{images/figures/newplot (28).png}
}
\caption{Regresja Lasso}
\end{figure}


\clearpage
\begin{figure}[ht]
\centering
\subfigure[Moduł sprężystości objętościowej]{%
    \includegraphics[width=0.48\textwidth]{images/figures/newplot (2).png}
}
\subfigure[Moduł sprężystości poprzecznej]{%
    \includegraphics[width=0.48\textwidth]{images/figures/newplot (11).png}
}
\\
\subfigure[Współczynninik Anizotropii]{%
    \includegraphics[width=0.48\textwidth]{images/figures/newplot (20).png}
}
\subfigure[Liczba Poisona]{%
    \includegraphics[width=0.48\textwidth]{images/figures/newplot (29).png}
}
\caption{Regresja Grzbietowa}
\end{figure}



\clearpage
\begin{figure}[ht]
\centering
\subfigure[Moduł sprężystości objętościowej]{%
    \includegraphics[width=0.48\textwidth]{images/figures/newplot (3).png}
}
\subfigure[Moduł sprężystości poprzecznej]{%
    \includegraphics[width=0.48\textwidth]{images/figures/newplot (12).png}
}
\\
\subfigure[Współczynninik Anizotropii]{%
    \includegraphics[width=0.48\textwidth]{images/figures/newplot (21).png}
}
\subfigure[Liczba Poisona]{%
    \includegraphics[width=0.48\textwidth]{images/figures/newplot (30).png}
}
\caption{Elastic Net}
\end{figure}



\clearpage
\begin{figure}[ht]
\centering
\subfigure[Moduł sprężystości objętościowej]{%
    \includegraphics[width=0.48\textwidth]{images/figures/newplot (4).png}
}
\subfigure[Moduł sprężystości poprzecznej]{%
    \includegraphics[width=0.48\textwidth]{images/figures/newplot (13).png}
}
\\
\subfigure[Współczynninik Anizotropii]{%
    \includegraphics[width=0.48\textwidth]{images/figures/newplot (22).png}
}
\subfigure[Liczba Poisona]{%
    \includegraphics[width=0.48\textwidth]{images/figures/newplot (31).png}
}
\caption{Regresja Huber'a}
\end{figure}


podsumowanie liniowych

\clearpage
\begin{figure}[ht]
\centering
\subfigure[Moduł sprężystości objętościowej]{%
    \includegraphics[width=0.48\textwidth]{images/figures/newplot (5).png}
}
\subfigure[Moduł sprężystości poprzecznej]{%
    \includegraphics[width=0.48\textwidth]{images/figures/newplot (14).png}
}
\\
\subfigure[Współczynninik Anizotropii]{%
    \includegraphics[width=0.48\textwidth]{images/figures/newplot (23).png}
}
\subfigure[Liczba Poisona]{%
    \includegraphics[width=0.48\textwidth]{images/figures/newplot (32).png}
}
\caption{Regresja liniowego SVR}
\end{figure}


opis

\clearpage
\begin{figure}[ht]
\centering
\subfigure[Moduł sprężystości objętościowej]{%
    \includegraphics[width=0.48\textwidth]{images/figures/newplot (6).png}
}
\subfigure[Moduł sprężystości poprzecznej]{%
    \includegraphics[width=0.48\textwidth]{images/figures/newplot (15).png}
}
\\
\subfigure[Współczynninik Anizotropii]{%
    \includegraphics[width=0.48\textwidth]{images/figures/newplot (24).png}
}
\subfigure[Liczba Poisona]{%
    \includegraphics[width=0.48\textwidth]{images/figures/newplot (33).png}
}
\caption{Regresja Theil-Sen}
\end{figure}

\clearpage
\begin{figure}[ht]
\centering
\subfigure[Moduł sprężystości objętościowej]{%
    \includegraphics[width=0.48\textwidth]{images/figures/newplot (7).png}
}
\subfigure[Moduł sprężystości poprzecznej]{%
    \includegraphics[width=0.48\textwidth]{images/figures/newplot (16).png}
}
\\
\subfigure[Współczynninik Anizotropii]{%
    \includegraphics[width=0.48\textwidth]{images/figures/newplot (25).png}
}
\subfigure[Liczba Poisona]{%
    \includegraphics[width=0.48\textwidth]{images/figures/newplot (34).png}
}
\caption{Regresor Drzewa Decyzyjnego}
\end{figure}

opis

\clearpage
\begin{figure}[ht]
\centering
\subfigure[Moduł sprężystości objętościowej]{%
    \includegraphics[width=0.48\textwidth]{images/figures/newplot (8).png}
}
\subfigure[Moduł sprężystości poprzecznej]{%
    \includegraphics[width=0.48\textwidth]{images/figures/newplot (16).png}
}
\\
\subfigure[Współczynninik Anizotropii]{%
    \includegraphics[width=0.48\textwidth]{images/figures/newplot (25).png}
}
\subfigure[Liczba Poisona]{%
    \includegraphics[width=0.48\textwidth]{images/figures/newplot (34).png}
}
\caption{Regresor Losowego Lasu}
\end{figure}



% \clearpage



\begin{table}[]
    \centering
    \resizebox{\textwidth}{!}{
   

    
    \begin{tabular}{llrrr}
    
    \toprule
     &  & R^2 [-] & RMSE [GPa | -] & czas wykonania[min] \\
    feature & model &  &  &  \\
    \midrule
    \multirow[t]{9}{*}{bulk\_modulus} & LinearRegression & 0.909 & 18.787 & 0.100 \\
     & Lasso  & 0.910 & 18.756 & 0.080 \\
     & Ridge  & 0.910 & 18.753 & 1.150 \\
     & ElasticNet  & 0.911 & 18.627 & 1.030 \\
     & HuberRegressor  & 0.910 & 18.748 & 2.230 \\
     & LinearSVR  & 0.909 & 18.776 & 5.320 \\
     & TheilSenRegressor  & 0.408 & 48.010 & 6.060 \\
     & DecisionTreeRegressor  & 0.551 & 41.814 & 3.990 \\
     & RandomForestRegressor  & 0.822 & 26.330 & 98.590 \\
    \cline{1-5}
    \multirow[t]{9}{*}{shear\_modulus} & LinearRegression & 0.830 & 15.766 & 0.020 \\
     & Lasso & 0.834 & 15.587 & 0.070 \\
     & Ridge & 0.831 & 15.733 & 1.050 \\
     & ElasticNet & 0.835 & 15.556 & 1.110 \\
     & HuberRegressor & 0.833 & 15.633 & 2.040 \\
     & LinearSVR & 0.827 & 15.906 & 11.110 \\
     & TheilSenRegressor & 0.005 & 38.184 & 6.800 \\
     & DecisionTreeRegressor & 0.545 & 25.824 & 5.870 \\
     & RandomForestRegressor & 0.800 & 17.107 & 101.640 \\
    \cline{1-5}
    \multirow[t]{9}{*}{universal\_anisotropy} & LinearRegression & 0.082 & 0.650 & 0.020 \\
     & Lasso & 0.140 & 0.629 & 0.060 \\
     & Ridge & 0.109 & 0.640 & 1.680 \\
     & ElasticNet & 0.146 & 0.627 & 0.860 \\
     & HuberRegressor & 0.062 & 0.657 & 2.100 \\
     & LinearSVR & 0.068 & 0.655 & 72.390 \\
     & TheilSenRegressor & -3.139 & 1.380 & 3.590 \\
     & DecisionTreeRegressor & 0.045 & 0.663 & 3.960 \\
     & RandomForestRegressor & 0.180 & 0.614 & 96.480 \\
    \cline{1-5}
    \multirow[t]{9}{*}{homogeneous\_poisson} & LinearRegression & 0.492 & 0.036 & 0.010 \\
     & Lasso & 0.502 & 0.035 & 0.040 \\
     & Ridge & 0.493 & 0.036 & 0.970 \\
     & ElasticNet & 0.502 & 0.035 & 0.440 \\
     & HuberRegressor & 0.488 & 0.036 & 2.230 \\
     & LinearSVR & 0.493 & 0.036 & 25.960 \\
     & TheilSenRegressor & -2.024 & 0.087 & 3.920 \\
     & DecisionTreeRegressor & 0.264 & 0.043 & 3.750 \\
     & RandomForestRegressor & 0.539 & 0.034 & 97.040 \\
    \cline{1-5}
    \bottomrule
    \end{tabular}
    
    
}

 \caption{Zbiór wszystkich wyników. Kolumna z wynikami RMSE posiada połączoną jednostkę dla wszystkich wyników. Moduły mają jednostkę [GPa], natomiast liczba poisson'a oraz współczynnik anizotropii są bezwymiarowe.}
    
\end{table}

\clearpage
