% Chapter 1
\phantomsection
\setstretch{1.5}
\justify
\fontsize{14}{16}\selectfont
\setlength{\parindent}{0pt}
% \setcounter{chapter}{1}
\chapter*{Cel pracy} 
% \pagenumbering{arabic} 
\label{chap:wstep}
\addcontentsline{toc}{chapter}{\textnormal{Cel pracy}}
\fontsize{12}{14}\selectfont
% \vspace{\baselineskip} 

Celem tej pracy inżynierskiej jest zaprezentowanie możliwości zastosowania uczenia maszynowego w przewidywaniu właściwości materiałów. Projekt ma na celu zbadanie, jak uczenie maszynowe może wspierać inżynierów materiałowych w identyfikacji i projektowaniu nowych materiałów. W ramach pracy zostanie przedstawiony proces zbierania danych z bazy Materials Project oraz ich przetwarzania w celu generowania cech niezbędnych do modelowania. Dodatkowo, przeprowadzona zostanie analiza skuteczności różnych modeli uczenia maszynowego w przewidywaniu właściwości materiałów, a wyniki zostaną wizualizowane dla lepszego zrozumienia i interpretacji.


\clearpage